\documentclass[10pt,a4paper]{article}
\usepackage[utf8]{inputenc}
\usepackage[english,russian]{babel} 
\usepackage{amsmath} 
\usepackage{amsfonts}
\usepackage{amssymb}
\usepackage[all]{xy}


\begin{document}

\newtheorem{notion}{Замечание}
\newtheorem{exercise}{Упражнение}[section]
\newtheorem{theorem}{Теорема}[section]
\newtheorem{proposition}{Утверждение}[section]
\newtheorem{definition}{Определение}[section]

\newcommand{\Js}{\mathbb{J}}
\newcommand{\Real}{\mathbb{R}}
\newcommand{\Comp}{\mathbb{C}}
\newcommand{\Nat}{\mathbb{N}}

\newcommand{\Jsq}{\hat{\mathbb{J}}}


\newcommand{\Hs}{\mathcal{H}}
\newcommand{\Is}{\mathbb{I}}




\newcommand{\Hq}{\hat{\mathcal{H}}}
\newcommand{\Eq}{\hat{E}}

\newcommand{\Kq}{\mathcal{K}}
\newcommand{\Iq}{\mathbb{I}}
\newcommand{\Dq}{\mathbb{D}}

\newcommand{\Ii}{\operatorname{I}}
\newcommand{\Pbr}[2]{ \{ {#1},{#2}\} }
\newcommand{\Com}[2]{\frac{1}{i\hbar}[{#1},{#2}]}
\newcommand{\Tr}[1]{\operatorname{Tr}{#1}}
\newcommand{\Id}{\operatorname{Id}}
\newcommand{\Dual}[2]{<{#1},{#2}>}
\newcommand{\Abs}[1]{\vert{#1}\vert}


\[
\Hs
\Jsq
\Hq_n
\Kq\Iq\Dq
\Pbr{f}{g}
\Com{A}{B}
\hbar
\Tr{A}
\Dual{x}{y}
\]
\section{Общие определения}

В класической механнике \cite{Arnold} представленны пары $(\Hs,E)$ - где $E$ - вещественное или комплексное векторное пространство четной вещественной размерности, называемое фазовым, $\Hs$ - алгебра бесконечно дифференцируемых действительнозначных функций на $E$, с операцией умножения определенной как перемножение функций. Алгебра $\Hs$ называется алгеброй наблюдаемых и снабженнается дополнительной операцией - скобкой Пуассона, определяющей динамику наблюдаемых.

\begin{definition}
Скобкой Пуасона на алгебре $\Hs$ называется операция, удволетворяющая следующим свойствам 
\begin{enumerate}
\item Антисимметричность
\[\Pbr{f}{g}=-\Pbr{g}{f}\]
\item Линейность по первому аргументу
\[\Pbr{\alpha g+\beta h}{f}=\alpha\Pbr{g}{f}+\beta\Pbr{h}{f}\]
\item Правило Лейбница
\[\Pbr{fg}{h}=\Pbr{f}{h}g+f\Pbr{g}{h}\]
\item Тождество Якоби
\[\Pbr{f}{\Pbr{g}{h}}+\Pbr{g}{\Pbr{h}{f}}+\Pbr{h}{\Pbr{f}{g}}\]
\end{enumerate}
для любых $f,g,h\in\Hs$.
\end{definition} 

В слуаче если фазовое пространство $E$ - бесконечномерное локально-выпуклое пространство, очевидным способом определения скобки Пуасона является введение на $E$ симплектической структуры\cite{RatjuSmolyanov}.
\begin{definition}
Симплектическим локально выпуклым пространством называется пара $(E,\mathbb{J} )$, где $E$ - локально выпуклое пространство, $\Js\colon E' \to E  $  - непрерывное линейное отображение, такое что выполнятся равенство $\mathbb J^*=-\mathbb{J}$.
\end{definition}

Пусть $\Hs$ снабжена слабой локально выпуклой топологией, в которой оператор взятия производной непрерывен, тогда положим:
\[\Pbr{f}{g}= \Dual{f'}{\Js g'}\]

Скобка Пуасона будет корректно определена, например, в случае, например,когда топология на $\Hs$ задается системой полунорм:
\[p_{C, n}(f)=\max_{x\in C}\Abs{f^{(n)}(x)} \]
где $C$ - компакт в $E$. Алгебру наблюдаемых, снабженную операцией скобки Пуасона будем называть Пуасоновой алгеброй.
\begin{definition}
Системой Гамильтона-Дирака на фазовом пространстве $E$ это набор $(E,\Js,\Hs,h,\Phi)$, где $(E,\Js)$ - образуют симплектическое локально выпуклое пространство, $ \Hs $ - агебра Пуасона с выделенным элементом $h$-Гамильтонианом,
 $\Phi \colon E\to E_\Phi$ - функцией связей.
\end{definition}
Множество $\Gamma_\Phi=\{x\in E,\Phi (x)=0\}$ будем называть поверхностью связей. Для системы $(E,\Js,\Hs,h,\Phi)$ рассмотрим расширенние гамильтониана:
\[h(x,\lambda)_\lambda=h(x)+\Dual{\Phi(x)}{\lambda}\]

\section{Динамика cистем Гамильтона-Дирака}

Пусть дана система Гамильтона-Дирака $(E,\Js,\Hs,h,\Phi)$. Её динамику можно задать двумя различными способами. В первом случае рассматривается эволюция наблюдаемых системы:
\[\dot{f(t)}=\Pbr{f(t)}{h_\lambda},\qquad f\in\Hs\]



Во втором случае рассматриваются траетории на фазовом пространстве. Будем говорить, что $x(t)$ - задает траекторию системы гамильтона дирака, если найдется $\lambda\colon E_\Phi'\to\Real$, что выполняется система уравнений Гамильтона-Дирака:

\[\dot{x}(t)=\Js h_\lambda'(x(t),\lambda(t)),\qquad \Phi (x(t))=0\]

В этом уранении состояние системы в каждый момент времени описывается вектором $(x(t),\lambda(t))$ из $E\times E_\Phi'$. Можно ввести понятие обобщенного состояния описываемого элементом $\Hs'$.

\begin{definition}
Сопряженное к $\Hs$ как к векторному пространству, будем называть множеством состояний системы Гамильтона-Дирака. Отображение $\psi\colon\Real\to\Hs'$ будем называть обобщенной траекторией системы Гамильтона-Дирака.
\end{definition}

Эволюцию обощенной траектории определим уравнением, двойственным к уравнению эволюции наблюдаемых. Зададим оператор дифференцирования $D_{h_\lambda}\colon\Hs\times E_\Phi'\to\Hs$ по формуле:
\[D_{h_\lambda}(\lambda) f(x)=\Pbr{f'(x)}{h_\lambda'(x,\lambda)}=\Dual{f'(x)}{\Js h_\lambda'(x,\lambda)}\]
тогда эволюция наблюдаемых описывается уравнением
\[\dot{f}= D_{h_\lambda} f\]

\begin{definition}
Обобщенным уравнением Гамильтона-Дирака будем называть следуещее:
\[\psi(t)={D_{h_\lambda}}^*(\lambda)\psi(t)\qquad\psi(t)\in\Hs'\]
где ${D_{h_\lambda}}^*$  сопряженное к ${D_{h_\lambda}}$ линейное отображение.
\end{definition}

Применяя естественное вложение $E$ в $\Hs'$, получаем из обобщенного уравнения Гамильтона-Дирака траекторное уравнеие Гамильтона-Дирака. 
\section{Квантование топологических пуасоновых алгебр}

Для задания процедуры квантования пары $(\Hs,E)$ необходимо определить следующие объекты и их свойства:
\begin{enumerate}

\item $(\Ii,\leq)$ - множество индексов с определенным на нем отношением предпорядка

\item $\Hq_n$ - набор топологических алгебр, проиндексированных элементами $n\in\Is$. Причём все $\Hq_n$ как векторые пространства обладают слабой топологией.

\item $\Kq_n\colon\Hs\to\Hq_n,\ n\in \Ii$ - непрерывные линенйные сюрьекции, называемые частичными квантованиями. $\Kq_n$ не обязательно является гомоморфизмом алгебр, но выполняется следуещее:

\[\Kq_n \Pbr{f}{g}=\Com{\Kq_n f}{\Kq_n g}\]

где $f,g\in\Hs$, $\hbar$ - действительная константа, а $[\cdot,\cdot]$ обозночает коммутатор элементов $\Hq_n$
\item $\Iq_{n,m}\colon\Hq_n\to\Hq_m,\ n\leq m,\ n,m\in\Ii$ - набор  гомоморфизмов алгебр, причём:
\[\Iq_{m,n}\circ\Iq_{n,k}=\Iq_{m,k}\]
для любых $n,m,k\in\Ii$, таких что $m\leq n\leq k$.

\item $\Dq_{n,m}\colon\Hq_n\to\Hq_m,\ n\geq m,\ n,m\in\Ii$ - набор  гомоморфизмов алгебр, таких что:
\[\Dq_{m,n}\circ\Dq_{n,k}=\Dq_{m,k}\]
для любых $n,m,k\in\Ii$, таких что $m\geq n\geq k$.

\item Для любых $n,m\in\Ii$ выполняется:
\[\Dq_{n,m}\circ\Iq_{m,n} = \Iq_{n,m}\circ\Dq_{m,n} = \Id_{\Hq_n}\]

\end{enumerate}

Гомоморфизмы $\Iq_n,\Dq_n$ и сопряженные как отоброжения линейных пространств к ним непрерывные линейные операторы $\Iq_n',\ \Dq_n'$ задают структуры, используя их становится возможным взятие проективного ($\varprojlim$) и индуктивного ($\varinjlim$) предела алгебр $\Hq_n$ и пространств сосояний $ \Hq_n' $.
\begin{definition}
Квантованием  $(\Hs,E)$ называется пара $(\Hq,\Eq)$, где $\Hq$ определяется как проективный предел алгебр $\Hq=\varprojlim \Hq_n$, а $\Eq$ определяется как индуктивнsq предел $\Eq=\varinjlim\Hq_n'$.
\end{definition}

Из определения проективного и индуктивных пределов следует, что существует непрерывое вложение фазового пространства $\Eq$ в пространство обобщеных состояний $\Hq'$. Будем рассматривать $\Hq$ как функции над $\Eq$, вычисляемые по закону
\[\hat{F}(\hat{x})=\Dual{\hat{x}}{\hat{F}},\qquad\hat{x}\in\Eq.\]


\section{Мера Вигнера и уравнения Мояла-Дирака}

Пусть $(\Hq,\Eq)$ результат квантования пары $(\Hs,E)$.
\begin{definition}
Мерой Вигнера, сопоставленной элементу фазового пространства $\hat{x}\in \Eq$ называется такой $W_x\in \Hs'$, что 
\[\Dual{\hat{F}}{\hat{x}}=\Dual{F}{W_x}\]
где  $\hat{F}$ - результат квантования наблюдаемой $F$, $\hat{F}=\Kq F$.
\end{definition}

Далее описывается динамика меры Вигнера для квантовых систем Гамильтона - Дирака.

\section{Пример: квантование по Вейлю конечномерной гамильтоновой системы}
\begin{center}
\emph{inverse limit},
\newline \emph{projective limit},
\newline 
$\varprojlim$,
\newline обратный предел,
\newline проективный предел
\\
$\xymatrix{
 & \bullet\ar@{-->}[d]^{\exists!}\ar@/^1em/[ddr]\ar@/_1em/[ddl] & \\
 & X\ar[dr]^{f_j}\ar[dl]_{f_i} & \\
X_i & & X_j\ar[ll]^{f_i^j}
}$ 
\end{center}\begin{center}

\emph{direct limit},
\newline \emph{injective limit},
\newline $\varinjlim$,
\newline прямой предел,
\newline инъективный предел
\\
$\xymatrix{
X_i\ar[rr]^{f_j^i}\ar[dr]_{f^i}\ar@/_1em/[ddr] & & X_j\ar[dl]^{f^j}\ar@/^1em/[ddl] \\
 & X\ar@{-->}[d]^{\exists!} & \\
 & \bullet &
}$ \\
\end{center}




\begin{thebibliography}{9}
\bibitem{Arnold}
В.И. Арнольд
\textit{Математические методы классической механики},
5-354-00341-5 , 2003 г.
\bibitem{RatjuSmolyanov}
Т.С. Ратью , О.Г. Смолянов
\textit{КВАНТОВАНИЕ ПО ВИГНЕРУ СИСТЕМ ГАМИЛЬТОНА–ДИРАКА.}
ДОКЛАДЫ АКАДЕМИИ НАУК, 2015, том 460, No 5, с. 525–528

\end{thebibliography}


\end{document}
